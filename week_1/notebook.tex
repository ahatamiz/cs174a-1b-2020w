
% Default to the notebook output style

    


% Inherit from the specified cell style.




    
\documentclass[11pt]{article}

    
    
    \usepackage[T1]{fontenc}
    % Nicer default font (+ math font) than Computer Modern for most use cases
    \usepackage{mathpazo}

    % Basic figure setup, for now with no caption control since it's done
    % automatically by Pandoc (which extracts ![](path) syntax from Markdown).
    \usepackage{graphicx}
    % We will generate all images so they have a width \maxwidth. This means
    % that they will get their normal width if they fit onto the page, but
    % are scaled down if they would overflow the margins.
    \makeatletter
    \def\maxwidth{\ifdim\Gin@nat@width>\linewidth\linewidth
    \else\Gin@nat@width\fi}
    \makeatother
    \let\Oldincludegraphics\includegraphics
    % Set max figure width to be 80% of text width, for now hardcoded.
    \renewcommand{\includegraphics}[1]{\Oldincludegraphics[width=.8\maxwidth]{#1}}
    % Ensure that by default, figures have no caption (until we provide a
    % proper Figure object with a Caption API and a way to capture that
    % in the conversion process - todo).
    \usepackage{caption}
    \DeclareCaptionLabelFormat{nolabel}{}
    \captionsetup{labelformat=nolabel}

    \usepackage{adjustbox} % Used to constrain images to a maximum size 
    \usepackage{xcolor} % Allow colors to be defined
    \usepackage{enumerate} % Needed for markdown enumerations to work
    \usepackage{geometry} % Used to adjust the document margins
    \usepackage{amsmath} % Equations
    \usepackage{amssymb} % Equations
    \usepackage{textcomp} % defines textquotesingle
    % Hack from http://tex.stackexchange.com/a/47451/13684:
    \AtBeginDocument{%
        \def\PYZsq{\textquotesingle}% Upright quotes in Pygmentized code
    }
    \usepackage{upquote} % Upright quotes for verbatim code
    \usepackage{eurosym} % defines \euro
    \usepackage[mathletters]{ucs} % Extended unicode (utf-8) support
    \usepackage[utf8x]{inputenc} % Allow utf-8 characters in the tex document
    \usepackage{fancyvrb} % verbatim replacement that allows latex
    \usepackage{grffile} % extends the file name processing of package graphics 
                         % to support a larger range 
    % The hyperref package gives us a pdf with properly built
    % internal navigation ('pdf bookmarks' for the table of contents,
    % internal cross-reference links, web links for URLs, etc.)
    \usepackage{hyperref}
    \usepackage{longtable} % longtable support required by pandoc >1.10
    \usepackage{booktabs}  % table support for pandoc > 1.12.2
    \usepackage[inline]{enumitem} % IRkernel/repr support (it uses the enumerate* environment)
    \usepackage[normalem]{ulem} % ulem is needed to support strikethroughs (\sout)
                                % normalem makes italics be italics, not underlines
    

    
    
    % Colors for the hyperref package
    \definecolor{urlcolor}{rgb}{0,.145,.698}
    \definecolor{linkcolor}{rgb}{.71,0.21,0.01}
    \definecolor{citecolor}{rgb}{.12,.54,.11}

    % ANSI colors
    \definecolor{ansi-black}{HTML}{3E424D}
    \definecolor{ansi-black-intense}{HTML}{282C36}
    \definecolor{ansi-red}{HTML}{E75C58}
    \definecolor{ansi-red-intense}{HTML}{B22B31}
    \definecolor{ansi-green}{HTML}{00A250}
    \definecolor{ansi-green-intense}{HTML}{007427}
    \definecolor{ansi-yellow}{HTML}{DDB62B}
    \definecolor{ansi-yellow-intense}{HTML}{B27D12}
    \definecolor{ansi-blue}{HTML}{208FFB}
    \definecolor{ansi-blue-intense}{HTML}{0065CA}
    \definecolor{ansi-magenta}{HTML}{D160C4}
    \definecolor{ansi-magenta-intense}{HTML}{A03196}
    \definecolor{ansi-cyan}{HTML}{60C6C8}
    \definecolor{ansi-cyan-intense}{HTML}{258F8F}
    \definecolor{ansi-white}{HTML}{C5C1B4}
    \definecolor{ansi-white-intense}{HTML}{A1A6B2}

    % commands and environments needed by pandoc snippets
    % extracted from the output of `pandoc -s`
    \providecommand{\tightlist}{%
      \setlength{\itemsep}{0pt}\setlength{\parskip}{0pt}}
    \DefineVerbatimEnvironment{Highlighting}{Verbatim}{commandchars=\\\{\}}
    % Add ',fontsize=\small' for more characters per line
    \newenvironment{Shaded}{}{}
    \newcommand{\KeywordTok}[1]{\textcolor[rgb]{0.00,0.44,0.13}{\textbf{{#1}}}}
    \newcommand{\DataTypeTok}[1]{\textcolor[rgb]{0.56,0.13,0.00}{{#1}}}
    \newcommand{\DecValTok}[1]{\textcolor[rgb]{0.25,0.63,0.44}{{#1}}}
    \newcommand{\BaseNTok}[1]{\textcolor[rgb]{0.25,0.63,0.44}{{#1}}}
    \newcommand{\FloatTok}[1]{\textcolor[rgb]{0.25,0.63,0.44}{{#1}}}
    \newcommand{\CharTok}[1]{\textcolor[rgb]{0.25,0.44,0.63}{{#1}}}
    \newcommand{\StringTok}[1]{\textcolor[rgb]{0.25,0.44,0.63}{{#1}}}
    \newcommand{\CommentTok}[1]{\textcolor[rgb]{0.38,0.63,0.69}{\textit{{#1}}}}
    \newcommand{\OtherTok}[1]{\textcolor[rgb]{0.00,0.44,0.13}{{#1}}}
    \newcommand{\AlertTok}[1]{\textcolor[rgb]{1.00,0.00,0.00}{\textbf{{#1}}}}
    \newcommand{\FunctionTok}[1]{\textcolor[rgb]{0.02,0.16,0.49}{{#1}}}
    \newcommand{\RegionMarkerTok}[1]{{#1}}
    \newcommand{\ErrorTok}[1]{\textcolor[rgb]{1.00,0.00,0.00}{\textbf{{#1}}}}
    \newcommand{\NormalTok}[1]{{#1}}
    
    % Additional commands for more recent versions of Pandoc
    \newcommand{\ConstantTok}[1]{\textcolor[rgb]{0.53,0.00,0.00}{{#1}}}
    \newcommand{\SpecialCharTok}[1]{\textcolor[rgb]{0.25,0.44,0.63}{{#1}}}
    \newcommand{\VerbatimStringTok}[1]{\textcolor[rgb]{0.25,0.44,0.63}{{#1}}}
    \newcommand{\SpecialStringTok}[1]{\textcolor[rgb]{0.73,0.40,0.53}{{#1}}}
    \newcommand{\ImportTok}[1]{{#1}}
    \newcommand{\DocumentationTok}[1]{\textcolor[rgb]{0.73,0.13,0.13}{\textit{{#1}}}}
    \newcommand{\AnnotationTok}[1]{\textcolor[rgb]{0.38,0.63,0.69}{\textbf{\textit{{#1}}}}}
    \newcommand{\CommentVarTok}[1]{\textcolor[rgb]{0.38,0.63,0.69}{\textbf{\textit{{#1}}}}}
    \newcommand{\VariableTok}[1]{\textcolor[rgb]{0.10,0.09,0.49}{{#1}}}
    \newcommand{\ControlFlowTok}[1]{\textcolor[rgb]{0.00,0.44,0.13}{\textbf{{#1}}}}
    \newcommand{\OperatorTok}[1]{\textcolor[rgb]{0.40,0.40,0.40}{{#1}}}
    \newcommand{\BuiltInTok}[1]{{#1}}
    \newcommand{\ExtensionTok}[1]{{#1}}
    \newcommand{\PreprocessorTok}[1]{\textcolor[rgb]{0.74,0.48,0.00}{{#1}}}
    \newcommand{\AttributeTok}[1]{\textcolor[rgb]{0.49,0.56,0.16}{{#1}}}
    \newcommand{\InformationTok}[1]{\textcolor[rgb]{0.38,0.63,0.69}{\textbf{\textit{{#1}}}}}
    \newcommand{\WarningTok}[1]{\textcolor[rgb]{0.38,0.63,0.69}{\textbf{\textit{{#1}}}}}
    
    
    % Define a nice break command that doesn't care if a line doesn't already
    % exist.
    \def\br{\hspace*{\fill} \\* }
    % Math Jax compatability definitions
    \def\gt{>}
    \def\lt{<}
    % Document parameters
    \title{cs174A-dis1B-week0}
    
    
    

    % Pygments definitions
    
\makeatletter
\def\PY@reset{\let\PY@it=\relax \let\PY@bf=\relax%
    \let\PY@ul=\relax \let\PY@tc=\relax%
    \let\PY@bc=\relax \let\PY@ff=\relax}
\def\PY@tok#1{\csname PY@tok@#1\endcsname}
\def\PY@toks#1+{\ifx\relax#1\empty\else%
    \PY@tok{#1}\expandafter\PY@toks\fi}
\def\PY@do#1{\PY@bc{\PY@tc{\PY@ul{%
    \PY@it{\PY@bf{\PY@ff{#1}}}}}}}
\def\PY#1#2{\PY@reset\PY@toks#1+\relax+\PY@do{#2}}

\expandafter\def\csname PY@tok@w\endcsname{\def\PY@tc##1{\textcolor[rgb]{0.73,0.73,0.73}{##1}}}
\expandafter\def\csname PY@tok@c\endcsname{\let\PY@it=\textit\def\PY@tc##1{\textcolor[rgb]{0.25,0.50,0.50}{##1}}}
\expandafter\def\csname PY@tok@cp\endcsname{\def\PY@tc##1{\textcolor[rgb]{0.74,0.48,0.00}{##1}}}
\expandafter\def\csname PY@tok@k\endcsname{\let\PY@bf=\textbf\def\PY@tc##1{\textcolor[rgb]{0.00,0.50,0.00}{##1}}}
\expandafter\def\csname PY@tok@kp\endcsname{\def\PY@tc##1{\textcolor[rgb]{0.00,0.50,0.00}{##1}}}
\expandafter\def\csname PY@tok@kt\endcsname{\def\PY@tc##1{\textcolor[rgb]{0.69,0.00,0.25}{##1}}}
\expandafter\def\csname PY@tok@o\endcsname{\def\PY@tc##1{\textcolor[rgb]{0.40,0.40,0.40}{##1}}}
\expandafter\def\csname PY@tok@ow\endcsname{\let\PY@bf=\textbf\def\PY@tc##1{\textcolor[rgb]{0.67,0.13,1.00}{##1}}}
\expandafter\def\csname PY@tok@nb\endcsname{\def\PY@tc##1{\textcolor[rgb]{0.00,0.50,0.00}{##1}}}
\expandafter\def\csname PY@tok@nf\endcsname{\def\PY@tc##1{\textcolor[rgb]{0.00,0.00,1.00}{##1}}}
\expandafter\def\csname PY@tok@nc\endcsname{\let\PY@bf=\textbf\def\PY@tc##1{\textcolor[rgb]{0.00,0.00,1.00}{##1}}}
\expandafter\def\csname PY@tok@nn\endcsname{\let\PY@bf=\textbf\def\PY@tc##1{\textcolor[rgb]{0.00,0.00,1.00}{##1}}}
\expandafter\def\csname PY@tok@ne\endcsname{\let\PY@bf=\textbf\def\PY@tc##1{\textcolor[rgb]{0.82,0.25,0.23}{##1}}}
\expandafter\def\csname PY@tok@nv\endcsname{\def\PY@tc##1{\textcolor[rgb]{0.10,0.09,0.49}{##1}}}
\expandafter\def\csname PY@tok@no\endcsname{\def\PY@tc##1{\textcolor[rgb]{0.53,0.00,0.00}{##1}}}
\expandafter\def\csname PY@tok@nl\endcsname{\def\PY@tc##1{\textcolor[rgb]{0.63,0.63,0.00}{##1}}}
\expandafter\def\csname PY@tok@ni\endcsname{\let\PY@bf=\textbf\def\PY@tc##1{\textcolor[rgb]{0.60,0.60,0.60}{##1}}}
\expandafter\def\csname PY@tok@na\endcsname{\def\PY@tc##1{\textcolor[rgb]{0.49,0.56,0.16}{##1}}}
\expandafter\def\csname PY@tok@nt\endcsname{\let\PY@bf=\textbf\def\PY@tc##1{\textcolor[rgb]{0.00,0.50,0.00}{##1}}}
\expandafter\def\csname PY@tok@nd\endcsname{\def\PY@tc##1{\textcolor[rgb]{0.67,0.13,1.00}{##1}}}
\expandafter\def\csname PY@tok@s\endcsname{\def\PY@tc##1{\textcolor[rgb]{0.73,0.13,0.13}{##1}}}
\expandafter\def\csname PY@tok@sd\endcsname{\let\PY@it=\textit\def\PY@tc##1{\textcolor[rgb]{0.73,0.13,0.13}{##1}}}
\expandafter\def\csname PY@tok@si\endcsname{\let\PY@bf=\textbf\def\PY@tc##1{\textcolor[rgb]{0.73,0.40,0.53}{##1}}}
\expandafter\def\csname PY@tok@se\endcsname{\let\PY@bf=\textbf\def\PY@tc##1{\textcolor[rgb]{0.73,0.40,0.13}{##1}}}
\expandafter\def\csname PY@tok@sr\endcsname{\def\PY@tc##1{\textcolor[rgb]{0.73,0.40,0.53}{##1}}}
\expandafter\def\csname PY@tok@ss\endcsname{\def\PY@tc##1{\textcolor[rgb]{0.10,0.09,0.49}{##1}}}
\expandafter\def\csname PY@tok@sx\endcsname{\def\PY@tc##1{\textcolor[rgb]{0.00,0.50,0.00}{##1}}}
\expandafter\def\csname PY@tok@m\endcsname{\def\PY@tc##1{\textcolor[rgb]{0.40,0.40,0.40}{##1}}}
\expandafter\def\csname PY@tok@gh\endcsname{\let\PY@bf=\textbf\def\PY@tc##1{\textcolor[rgb]{0.00,0.00,0.50}{##1}}}
\expandafter\def\csname PY@tok@gu\endcsname{\let\PY@bf=\textbf\def\PY@tc##1{\textcolor[rgb]{0.50,0.00,0.50}{##1}}}
\expandafter\def\csname PY@tok@gd\endcsname{\def\PY@tc##1{\textcolor[rgb]{0.63,0.00,0.00}{##1}}}
\expandafter\def\csname PY@tok@gi\endcsname{\def\PY@tc##1{\textcolor[rgb]{0.00,0.63,0.00}{##1}}}
\expandafter\def\csname PY@tok@gr\endcsname{\def\PY@tc##1{\textcolor[rgb]{1.00,0.00,0.00}{##1}}}
\expandafter\def\csname PY@tok@ge\endcsname{\let\PY@it=\textit}
\expandafter\def\csname PY@tok@gs\endcsname{\let\PY@bf=\textbf}
\expandafter\def\csname PY@tok@gp\endcsname{\let\PY@bf=\textbf\def\PY@tc##1{\textcolor[rgb]{0.00,0.00,0.50}{##1}}}
\expandafter\def\csname PY@tok@go\endcsname{\def\PY@tc##1{\textcolor[rgb]{0.53,0.53,0.53}{##1}}}
\expandafter\def\csname PY@tok@gt\endcsname{\def\PY@tc##1{\textcolor[rgb]{0.00,0.27,0.87}{##1}}}
\expandafter\def\csname PY@tok@err\endcsname{\def\PY@bc##1{\setlength{\fboxsep}{0pt}\fcolorbox[rgb]{1.00,0.00,0.00}{1,1,1}{\strut ##1}}}
\expandafter\def\csname PY@tok@kc\endcsname{\let\PY@bf=\textbf\def\PY@tc##1{\textcolor[rgb]{0.00,0.50,0.00}{##1}}}
\expandafter\def\csname PY@tok@kd\endcsname{\let\PY@bf=\textbf\def\PY@tc##1{\textcolor[rgb]{0.00,0.50,0.00}{##1}}}
\expandafter\def\csname PY@tok@kn\endcsname{\let\PY@bf=\textbf\def\PY@tc##1{\textcolor[rgb]{0.00,0.50,0.00}{##1}}}
\expandafter\def\csname PY@tok@kr\endcsname{\let\PY@bf=\textbf\def\PY@tc##1{\textcolor[rgb]{0.00,0.50,0.00}{##1}}}
\expandafter\def\csname PY@tok@bp\endcsname{\def\PY@tc##1{\textcolor[rgb]{0.00,0.50,0.00}{##1}}}
\expandafter\def\csname PY@tok@fm\endcsname{\def\PY@tc##1{\textcolor[rgb]{0.00,0.00,1.00}{##1}}}
\expandafter\def\csname PY@tok@vc\endcsname{\def\PY@tc##1{\textcolor[rgb]{0.10,0.09,0.49}{##1}}}
\expandafter\def\csname PY@tok@vg\endcsname{\def\PY@tc##1{\textcolor[rgb]{0.10,0.09,0.49}{##1}}}
\expandafter\def\csname PY@tok@vi\endcsname{\def\PY@tc##1{\textcolor[rgb]{0.10,0.09,0.49}{##1}}}
\expandafter\def\csname PY@tok@vm\endcsname{\def\PY@tc##1{\textcolor[rgb]{0.10,0.09,0.49}{##1}}}
\expandafter\def\csname PY@tok@sa\endcsname{\def\PY@tc##1{\textcolor[rgb]{0.73,0.13,0.13}{##1}}}
\expandafter\def\csname PY@tok@sb\endcsname{\def\PY@tc##1{\textcolor[rgb]{0.73,0.13,0.13}{##1}}}
\expandafter\def\csname PY@tok@sc\endcsname{\def\PY@tc##1{\textcolor[rgb]{0.73,0.13,0.13}{##1}}}
\expandafter\def\csname PY@tok@dl\endcsname{\def\PY@tc##1{\textcolor[rgb]{0.73,0.13,0.13}{##1}}}
\expandafter\def\csname PY@tok@s2\endcsname{\def\PY@tc##1{\textcolor[rgb]{0.73,0.13,0.13}{##1}}}
\expandafter\def\csname PY@tok@sh\endcsname{\def\PY@tc##1{\textcolor[rgb]{0.73,0.13,0.13}{##1}}}
\expandafter\def\csname PY@tok@s1\endcsname{\def\PY@tc##1{\textcolor[rgb]{0.73,0.13,0.13}{##1}}}
\expandafter\def\csname PY@tok@mb\endcsname{\def\PY@tc##1{\textcolor[rgb]{0.40,0.40,0.40}{##1}}}
\expandafter\def\csname PY@tok@mf\endcsname{\def\PY@tc##1{\textcolor[rgb]{0.40,0.40,0.40}{##1}}}
\expandafter\def\csname PY@tok@mh\endcsname{\def\PY@tc##1{\textcolor[rgb]{0.40,0.40,0.40}{##1}}}
\expandafter\def\csname PY@tok@mi\endcsname{\def\PY@tc##1{\textcolor[rgb]{0.40,0.40,0.40}{##1}}}
\expandafter\def\csname PY@tok@il\endcsname{\def\PY@tc##1{\textcolor[rgb]{0.40,0.40,0.40}{##1}}}
\expandafter\def\csname PY@tok@mo\endcsname{\def\PY@tc##1{\textcolor[rgb]{0.40,0.40,0.40}{##1}}}
\expandafter\def\csname PY@tok@ch\endcsname{\let\PY@it=\textit\def\PY@tc##1{\textcolor[rgb]{0.25,0.50,0.50}{##1}}}
\expandafter\def\csname PY@tok@cm\endcsname{\let\PY@it=\textit\def\PY@tc##1{\textcolor[rgb]{0.25,0.50,0.50}{##1}}}
\expandafter\def\csname PY@tok@cpf\endcsname{\let\PY@it=\textit\def\PY@tc##1{\textcolor[rgb]{0.25,0.50,0.50}{##1}}}
\expandafter\def\csname PY@tok@c1\endcsname{\let\PY@it=\textit\def\PY@tc##1{\textcolor[rgb]{0.25,0.50,0.50}{##1}}}
\expandafter\def\csname PY@tok@cs\endcsname{\let\PY@it=\textit\def\PY@tc##1{\textcolor[rgb]{0.25,0.50,0.50}{##1}}}

\def\PYZbs{\char`\\}
\def\PYZus{\char`\_}
\def\PYZob{\char`\{}
\def\PYZcb{\char`\}}
\def\PYZca{\char`\^}
\def\PYZam{\char`\&}
\def\PYZlt{\char`\<}
\def\PYZgt{\char`\>}
\def\PYZsh{\char`\#}
\def\PYZpc{\char`\%}
\def\PYZdl{\char`\$}
\def\PYZhy{\char`\-}
\def\PYZsq{\char`\'}
\def\PYZdq{\char`\"}
\def\PYZti{\char`\~}
% for compatibility with earlier versions
\def\PYZat{@}
\def\PYZlb{[}
\def\PYZrb{]}
\makeatother


    % Exact colors from NB
    \definecolor{incolor}{rgb}{0.0, 0.0, 0.5}
    \definecolor{outcolor}{rgb}{0.545, 0.0, 0.0}



    
    % Prevent overflowing lines due to hard-to-break entities
    \sloppy 
    % Setup hyperref package
    \hypersetup{
      breaklinks=true,  % so long urls are correctly broken across lines
      colorlinks=true,
      urlcolor=urlcolor,
      linkcolor=linkcolor,
      citecolor=citecolor,
      }
    % Slightly bigger margins than the latex defaults
    
    \geometry{verbose,tmargin=1in,bmargin=1in,lmargin=1in,rmargin=1in}
    
    

    \begin{document}
    
    
    \maketitle
    
    

    
    \section{CS-174A Discussion 1B, Week
0}\label{cs-174a-discussion-1b-week-0}

@ Ali Hatamizadeh

@ ROLFE 3126 / Friday / 2:00pm- 3:50pm

@ https://github.com/ahatamiz/cs174a-1b-2020w

    \section{Outline}\label{outline}

\begin{itemize}
\tightlist
\item
  About this course
\item
  JavaScript and WebGL Basic
\item
  Assignment 1
\end{itemize}

    \section{CS-174A Introduction}\label{cs-174a-introduction}

\subsection{About Me:}\label{about-me}

\begin{itemize}
\item
  Ali Hatamizadeh, Ph.D student in Computer Science
\item
  Office hours: Eng-VI 366, Friday 4:00 - 6:00 PM
\item
  Email: ahatamiz@ucla.edu
\end{itemize}

    \subsubsection{Grading Policies}\label{grading-policies}

\begin{itemize}
\tightlist
\item
  4 assignments (0 + 10 + 10 + 10): 30 pts
\item
  Team project: 30 pts
\item
  Midterm: 15 pts
\item
  Final: 25 pts
\end{itemize}

(May change. Stay tuned till the next lecture)

    \section{JavaScript Basics}\label{javascript-basics}

JavaScript can change HTML content

    \begin{Verbatim}[commandchars=\\\{\}]
{\color{incolor}In [{\color{incolor}1}]:} \PY{o}{\PYZpc{}\PYZpc{}}\PY{k}{html}
        \PYZlt{}p id=\PYZdq{}demo\PYZdq{}\PYZgt{}JavaScript can change HTML content.\PYZlt{}/p\PYZgt{}
        \PYZlt{}button type=\PYZdq{}button\PYZdq{} onclick=\PYZsq{}document.getElementById(\PYZdq{}demo\PYZdq{}).innerHTML = \PYZdq{}Hello JavaScript!\PYZdq{}\PYZsq{}\PYZgt{}Click Me!\PYZlt{}/button\PYZgt{}
\end{Verbatim}


    
    \begin{verbatim}
<IPython.core.display.HTML object>
    \end{verbatim}

    
    \subsection{Let vs Var vs Constant}\label{let-vs-var-vs-constant}

var: When you declare a variable with var, its scope is not limited to
the block in which it is defined. It's limited to the function in which
it is defined.

\begin{Shaded}
\begin{Highlighting}[]
\KeywordTok{function} \AttributeTok{start}\NormalTok{()}\OperatorTok{\{}
\ControlFlowTok{for}\NormalTok{ (}\KeywordTok{var}\NormalTok{ i}\OperatorTok{=}\DecValTok{0}\OperatorTok{;}\NormalTok{i}\OperatorTok{<}\DecValTok{5}\OperatorTok{;}\NormalTok{i}\OperatorTok{++}\NormalTok{)}\OperatorTok{\{}

\OperatorTok{\}}

\VariableTok{element}\NormalTok{.}\AttributeTok{text}\NormalTok{(i)}

\OperatorTok{\}}
\end{Highlighting}
\end{Shaded}

    \begin{Verbatim}[commandchars=\\\{\}]
{\color{incolor}In [{\color{incolor}2}]:} \PY{o}{\PYZpc{}\PYZpc{}}\PY{k}{js}
        
        function start()\PYZob{}
        for (var i=0;i\PYZlt{}5;i++)\PYZob{}
        \PYZcb{}
        
        element.text(i);
        
        
        \PYZcb{}
        
        start()
\end{Verbatim}


    
    \begin{verbatim}
<IPython.core.display.Javascript object>
    \end{verbatim}

    
    \subsection{Let vs Var vs Constant}\label{let-vs-var-vs-constant}

let and constant are block-scoped

\begin{Shaded}
\begin{Highlighting}[]
\KeywordTok{function} \AttributeTok{start}\NormalTok{()}\OperatorTok{\{}
\ControlFlowTok{for}\NormalTok{ (}\KeywordTok{let}\NormalTok{ i}\OperatorTok{=}\DecValTok{0}\OperatorTok{;}\NormalTok{i}\OperatorTok{<}\DecValTok{5}\OperatorTok{;}\NormalTok{i}\OperatorTok{++}\NormalTok{)}\OperatorTok{\{}
\OperatorTok{\}}

\VariableTok{element}\NormalTok{.}\AttributeTok{text}\NormalTok{(i)}

\OperatorTok{\}}
\end{Highlighting}
\end{Shaded}

    \begin{Verbatim}[commandchars=\\\{\}]
{\color{incolor}In [{\color{incolor}3}]:} \PY{o}{\PYZpc{}\PYZpc{}}\PY{k}{js}
        
        function start()\PYZob{}
        for (let i=0;i\PYZlt{}5;i++)\PYZob{}
        \PYZcb{}
        
        element.text(i)
        
        \PYZcb{}
        
        start()
\end{Verbatim}


    
    \begin{verbatim}
<IPython.core.display.Javascript object>
    \end{verbatim}

    
    \subsection{Some additional points}\label{some-additional-points}

When you use var outside of a function, it creates a global variable and
attaches it to the window object in the browser.

When you use let to create a global variable, it is not attached to the
window object.

    \subsection{Variables and Data types}\label{variables-and-data-types}

JavaScript variables are containers for storing data values.

JavaScript variables can hold many \textbf{data types}: numbers,
strings, objects and more:

\begin{Shaded}
\begin{Highlighting}[]
\KeywordTok{var}\NormalTok{ length }\OperatorTok{=} \DecValTok{16}\OperatorTok{;}                            \CommentTok{// Number}
\KeywordTok{var}\NormalTok{ lastName }\OperatorTok{=} \StringTok{"Johnson"}\OperatorTok{;}                   \CommentTok{// String}
\KeywordTok{var}\NormalTok{ x }\OperatorTok{=} \OperatorTok{\{}\DataTypeTok{firstName}\OperatorTok{:}\StringTok{"John"}\OperatorTok{,} \DataTypeTok{lastName}\OperatorTok{:}\StringTok{"Doe"}\OperatorTok{\};} \CommentTok{// Object}
\end{Highlighting}
\end{Shaded}

    \begin{Verbatim}[commandchars=\\\{\}]
{\color{incolor}In [{\color{incolor}4}]:} \PY{o}{\PYZpc{}\PYZpc{}}\PY{k}{js}
        var num = 16;                                  // Number
        element.text(\PYZdq{}The number is \PYZdq{} + num)
\end{Verbatim}


    
    \begin{verbatim}
<IPython.core.display.Javascript object>
    \end{verbatim}

    
    \begin{Verbatim}[commandchars=\\\{\}]
{\color{incolor}In [{\color{incolor}5}]:} \PY{o}{\PYZpc{}\PYZpc{}}\PY{k}{js}
        element.text(\PYZsq{}here we go\PYZsq{})
\end{Verbatim}


    
    \begin{verbatim}
<IPython.core.display.Javascript object>
    \end{verbatim}

    
    \subsection{Objects}\label{objects}

You define (and create) a JavaScript object with an object literal:

\begin{Shaded}
\begin{Highlighting}[]
\KeywordTok{var}\NormalTok{ person }\OperatorTok{=} \OperatorTok{\{}\DataTypeTok{firstName}\OperatorTok{:}\StringTok{"John"}\OperatorTok{,} \DataTypeTok{lastName}\OperatorTok{:}\StringTok{"Doe"}\OperatorTok{,} \DataTypeTok{age}\OperatorTok{:}\DecValTok{50}\OperatorTok{,} \DataTypeTok{eyeColor}\OperatorTok{:}\StringTok{"blue"}\OperatorTok{\};}
\end{Highlighting}
\end{Shaded}

    \subsection{Objects}\label{objects}

Another way of creating object You define (and create) a JavaScript
object with an object literal:

\begin{Shaded}
\begin{Highlighting}[]
\KeywordTok{var}\NormalTok{ person}\OperatorTok{=} \KeywordTok{new} \AttributeTok{Object}\NormalTok{()}
\VariableTok{person}\NormalTok{.}\AttributeTok{firstName}\OperatorTok{=}\StringTok{"John"}
\VariableTok{person}\NormalTok{.}\AttributeTok{age}\OperatorTok{=}\DecValTok{50}
\VariableTok{person}\NormalTok{.}\AttributeTok{eyeColor}\OperatorTok{=}\StringTok{"blue"}
\end{Highlighting}
\end{Shaded}

    \begin{Verbatim}[commandchars=\\\{\}]
{\color{incolor}In [{\color{incolor}6}]:} \PY{o}{\PYZpc{}\PYZpc{}}\PY{k}{js}
        var person= new Object()
        person.firstName=\PYZdq{}John\PYZdq{}
        person.age=50
        person.eyeColor=\PYZdq{}blue\PYZdq{}
        element.text(person[\PYZsq{}eyeColor\PYZsq{}])
\end{Verbatim}


    
    \begin{verbatim}
<IPython.core.display.Javascript object>
    \end{verbatim}

    
    \begin{Verbatim}[commandchars=\\\{\}]
{\color{incolor}In [{\color{incolor}7}]:} \PY{o}{\PYZpc{}\PYZpc{}}\PY{k}{js}
        var person = \PYZob{}firstName:\PYZdq{}John\PYZdq{}, lastName:\PYZdq{}Doe\PYZdq{}, age:50, eyeColor:\PYZdq{}blue\PYZdq{}\PYZcb{};
        element.text(person.firstName + \PYZdq{}\PYZsq{}s age is \PYZdq{} + person[\PYZdq{}age\PYZdq{}]);  // two ways for accessing the property of an object
        
        var name = new String(\PYZdq{}John\PYZdq{});
        var name\PYZus{}2 = \PYZdq{}John\PYZdq{};
        element.text(name === \PYZdq{}John\PYZdq{}); 
\end{Verbatim}


    
    \begin{verbatim}
<IPython.core.display.Javascript object>
    \end{verbatim}

    
    \subsection{Functions}\label{functions}

A JavaScript function is a block of code designed to perform a
particular task.

A JavaScript function is executed when "something" invokes it (calls
it).

    \begin{Verbatim}[commandchars=\\\{\}]
{\color{incolor}In [{\color{incolor}8}]:} \PY{o}{\PYZpc{}\PYZpc{}}\PY{k}{js}
        function myFunction(p1, p2) \PYZob{}
          return p1 * p2;   // The function returns the product of p1 and p2
        \PYZcb{}
        
        var a = 3;
        var b = 4;
        element.text(\PYZdq{}The product of a and b is \PYZdq{} + myFunction(a,b))
\end{Verbatim}


    
    \begin{verbatim}
<IPython.core.display.Javascript object>
    \end{verbatim}

    
    \subsection{Factory Functions}\label{factory-functions}

A factory function creates an object

\begin{Shaded}
\begin{Highlighting}[]
\KeywordTok{function} \AttributeTok{createCircle}\NormalTok{(radius}\OperatorTok{,}\NormalTok{location)}\OperatorTok{\{}
    
    \ControlFlowTok{return} \OperatorTok{\{}
        \DataTypeTok{radius}\OperatorTok{:}\NormalTok{radius}\OperatorTok{,}
        \DataTypeTok{location}\OperatorTok{:}\NormalTok{location}\OperatorTok{,}
        \DataTypeTok{visible}\OperatorTok{:}\KeywordTok{true}\OperatorTok{,}
        \DataTypeTok{draw}\OperatorTok{:} \KeywordTok{function}\NormalTok{()}\OperatorTok{\{}\VariableTok{element}\NormalTok{.}\AttributeTok{text}\NormalTok{(}\StringTok{'Here we go : draw'}\NormalTok{)}\OperatorTok{\}}
    \OperatorTok{\}}

\OperatorTok{\}}
\end{Highlighting}
\end{Shaded}

    \begin{Verbatim}[commandchars=\\\{\}]
{\color{incolor}In [{\color{incolor}9}]:} \PY{o}{\PYZpc{}\PYZpc{}}\PY{k}{js} 
        function createCircle(radius)\PYZob{}
            
            return \PYZob{}
                radius:radius,
                visible:true,
                draw: function()\PYZob{}element.text(\PYZsq{}draw\PYZsq{})\PYZcb{}
            \PYZcb{}
        
        \PYZcb{}
        
        const circle1=createCircle(1)
        circle1.draw()
\end{Verbatim}


    
    \begin{verbatim}
<IPython.core.display.Javascript object>
    \end{verbatim}

    
    \subsection{JavaScript this Keyword}\label{javascript-this-keyword}

" this " in JavaScript refers to the object that is executing the
current function

If the function is part of an object ( in other words is a method of
that object): " this " refrences the object itself

Otherwise, " this " refers to the global object ( which is window object
in browsers)

    \begin{Verbatim}[commandchars=\\\{\}]
{\color{incolor}In [{\color{incolor}10}]:} \PY{o}{\PYZpc{}\PYZpc{}}\PY{k}{js} 
         
         const video =\PYZob{}
             title:\PYZsq{}a\PYZsq{},
             play()\PYZob{}
                 element.text(this)
             \PYZcb{}
         \PYZcb{}
         
         video.play()
\end{Verbatim}


    
    \begin{verbatim}
<IPython.core.display.Javascript object>
    \end{verbatim}

    
    \begin{Verbatim}[commandchars=\\\{\}]
{\color{incolor}In [{\color{incolor}11}]:} \PY{o}{\PYZpc{}\PYZpc{}}\PY{k}{js} 
         
         const video =\PYZob{}
             title:\PYZsq{}a\PYZsq{},
             play()\PYZob{}
                 element.text(this)
             \PYZcb{}
         \PYZcb{}
         
         function playVideo()\PYZob{}
             element.text(this)
         \PYZcb{}
         
         playVideo()
\end{Verbatim}


    
    \begin{verbatim}
<IPython.core.display.Javascript object>
    \end{verbatim}

    
    \subsection{Constructor Functions}\label{constructor-functions}

A constructor function also creates an object. Be aware of the naming
conventions:

Camel notation: oneTwoThree ( Used for naming factory functions)

Pascal notation : OneTwoThree ( Used for naming constructor functions)

\begin{Shaded}
\begin{Highlighting}[]

\KeywordTok{function} \AttributeTok{Circle}\NormalTok{(radius}\OperatorTok{,}\NormalTok{location)}\OperatorTok{\{}
    \CommentTok{//this here is an empty object to which we add radius and location }
    \CommentTok{//JavaScripts objects are dynamic. Once created, we can add aditional methods to them.}
    \KeywordTok{this}\NormalTok{.}\AttributeTok{radius}\OperatorTok{=}\NormalTok{radius}
    \KeywordTok{this}\NormalTok{.}\AttributeTok{location}\OperatorTok{=}\NormalTok{location}
    \KeywordTok{this}\NormalTok{.}\AttributeTok{visible}\OperatorTok{=}\KeywordTok{true}
    \KeywordTok{this}\NormalTok{.}\AttributeTok{draw}\OperatorTok{=}\KeywordTok{function}\NormalTok{()}\OperatorTok{\{}\VariableTok{element}\NormalTok{.}\AttributeTok{text}\NormalTok{(}\StringTok{'Here we go : draw'}\NormalTok{)}\OperatorTok{\}}
    
\OperatorTok{\}}

\end{Highlighting}
\end{Shaded}

    \begin{Verbatim}[commandchars=\\\{\}]
{\color{incolor}In [{\color{incolor}12}]:} \PY{o}{\PYZpc{}\PYZpc{}}\PY{k}{js}
         
         function Circle(radius)\PYZob{}
             //this here is an empty object to which we add radius
             //JavaScripts objects are dynamic. Once created, we can add aditional methods to them.
             this.radius=radius
             this.visible=true
             this.draw=function()\PYZob{}element.text(\PYZsq{}Here we go : draw\PYZsq{})\PYZcb{}
             
         \PYZcb{}
         
         const circle1=new Circle(1)
         circle1.draw()
\end{Verbatim}


    
    \begin{verbatim}
<IPython.core.display.Javascript object>
    \end{verbatim}

    
    \subsection{Object Methods}\label{object-methods}

    \begin{Verbatim}[commandchars=\\\{\}]
{\color{incolor}In [{\color{incolor}13}]:} \PY{o}{\PYZpc{}\PYZpc{}}\PY{k}{js}
         var person = \PYZob{}
           firstName: \PYZdq{}John\PYZdq{},
           lastName : \PYZdq{}Doe\PYZdq{},
           id       : 5566,
           fullName : function() \PYZob{}
             return this.firstName + \PYZdq{} \PYZdq{} + this.lastName;
           \PYZcb{}
         \PYZcb{};
         element.text(person.fullName())
\end{Verbatim}


    
    \begin{verbatim}
<IPython.core.display.Javascript object>
    \end{verbatim}

    
    \subsubsection{Object Constructors}\label{object-constructors}

    \begin{Verbatim}[commandchars=\\\{\}]
{\color{incolor}In [{\color{incolor}14}]:} \PY{o}{\PYZpc{}\PYZpc{}}\PY{k}{js}
         function Person(first, last, age, eye) \PYZob{}
           this.firstName = first;
           this.lastName = last;
           this.age = age;
           this.eyeColor = eye;
         \PYZcb{}
         
         var myFather = new Person(\PYZdq{}John\PYZdq{}, \PYZdq{}Doe\PYZdq{}, 50, \PYZdq{}blue\PYZdq{});
         
         element.text(\PYZdq{}My father is \PYZdq{} + myFather.firstName + \PYZdq{} \PYZdq{} + myFather.lastName + \PYZdq{}.\PYZdq{})
         //element.text(myFather)
\end{Verbatim}


    
    \begin{verbatim}
<IPython.core.display.Javascript object>
    \end{verbatim}

    
    \subsection{JavaScript Classes}\label{javascript-classes}

A class is a type of function, but instead of using the keyword
\texttt{function} to initiate it, we use the keyword \texttt{class}, and
the properties is assigned inside a \texttt{constructor()} method.

    \subsubsection{Class Definition}\label{class-definition}

Use the keyword \texttt{class} to create a class, and always add a
\texttt{constructor} method.

The \texttt{constructor} method is called each time the class object is
initialized.

\begin{Shaded}
\begin{Highlighting}[]
\KeywordTok{class}\NormalTok{ Car }\OperatorTok{\{}
  \AttributeTok{constructor}\NormalTok{(brand) }\OperatorTok{\{}
    \KeywordTok{this}\NormalTok{.}\AttributeTok{carname} \OperatorTok{=}\NormalTok{ brand}\OperatorTok{;}
  \OperatorTok{\}}
\OperatorTok{\}}
\NormalTok{mycar }\OperatorTok{=} \KeywordTok{new} \AttributeTok{Car}\NormalTok{(}\StringTok{"Ford"}\NormalTok{)}\OperatorTok{;}
\end{Highlighting}
\end{Shaded}

    \subsection{A Quick Way to Learn JS}\label{a-quick-way-to-learn-js}

\begin{itemize}
\tightlist
\item
  https://www.w3schools.com/js/
\end{itemize}

    \section{WebGL Basics}\label{webgl-basics}

WebGL (Web Graphics Library) is a JavaScript API for rendering
interactive 3D and 2D graphics within any compatible web browser without
the use of plug-ins. WebGL does so by introducing an API that closely
conforms to OpenGL ES 2.0 that can be used in HTML5
\texttt{\textless{}canvas\textgreater{}} elements.

\begin{itemize}
\tightlist
\item
  \href{demo_0/demo.html}{Demo 0: Clearing with colors}. How to clear
  the rendering context with a solid color.
\item
  \href{demo_1/index.html}{Demo 1: Simple color animation}. A very basic
  color animation.
\end{itemize}

    \subsection{... and some other projects}\label{and-some-other-projects}

\begin{itemize}
\tightlist
\item
  \href{http://madebyevan.com/webgl-water/}{WebGL Wather}
\item
  \href{http://2017.makemepulse.com}{Make me pulse wish 2017}
\end{itemize}

    \section{Assignment 1: Environment Setup and Creating a Simple
Rectangle}\label{assignment-1-environment-setup-and-creating-a-simple-rectangle}

Posted on CCLE (Week 1)


    % Add a bibliography block to the postdoc
    
    
    
    \end{document}
